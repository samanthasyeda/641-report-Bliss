\section{Related Work}
\label{sec:related}
In early works on garbage collecting, Dijkstra \textit{et
al.}~\cite{dijkstra1978} has presented a graph marking algorithm that tranverses
computational graphs to discover the connectivity of the graph. One aspect of
these graph marking algorithms is that the connectivity of the computational
graph is constantly changing. A question that naturally rises is what graph
mutations can work execute concurrently with the marking process. In Dijkstra's
work, a sequential mutator running in parallel with a sequential garbage
collector has been shown to have to cooperate with each other to not invalidate
the marking process. In Hudak's work~\cite{Hudak:1983}
of garbage collecting in distrubuted computing systems, the problem has
been expanded to parallel mutators with parallel garbage collectors. Hudak
introduced a subset of mutation primitives and have shown that the basic set of
mutation primitives can cover most of the common mutations of a computational
graph, which can be accomodated with the garbage collector.

Strecker's attempt of formalizing graph
transformations~\cite{strecker2008modeling}, presented a more expressive set of 
operations for structural mutation of graphs. They have explored and defined the
applicability of graph transformations such that applying a well-formed graph
transformation on well-formed graphs can result in well-formed graphs. Our work
uses a similar approach to model the graph mutations of a graph, and
solves the problem of conflictions where a well-formed graph can not be derived
from a set of mutations.



